\documentclass[11pt]{article}
\usepackage{amsmath}
\usepackage{amssymb}
\usepackage{graphicx}
\usepackage{hyperref}
\providecommand{\e}[1]{\ensuremath{\times 10^{#1}}}
\DeclareMathOperator{\R}{\mathbb{R}}
\newcommand{\p}[1]{\partial_{#1}}

\begin{document}
\title{PETSc finite element details}
\author{Geoffrey Irving \\ Otherlab}
\maketitle

\section{Weak form PDEs}

Consider a strong form PDE with fields $u_0, u_1, \ldots$, each of which has components
$u_{i0}, u_{i1}, \ldots$, and equations
\begin{align*}
  f_{ab}\left(u_{ij}, \p{k} u_{ij}, \p{k}\p{l} u_{ij}\right) = 0
\end{align*}
where we have one $f_{ij}$ for each $u_{ij}$.  We use the convention that an index occurring
only inside a function call means that all values of the index are passed to the function.  When
we later discretize over space, $u_i$ and $u_j$ will be allowed to live in different discrete
function spaces, while $u_{ij}$ and $u_{ik}$ come from the same function space.

We pass to the weak form by integrating a smooth test function $v_{ab}$ against $f_{ab}$, then
integrating by parts to get
\begin{align*}
  \int_{\Omega} v_{ab} f_{ab} \left(u_{ij}, \p{k} u_{ij}, \p{k}\p{l} u_{ij} \right) dx &= 0 \\
  \int_{\Omega} v_{ab} f^0_{ab} \left(u_{ij}, \p{k} u_{ij} \right) dx
    + \int_{\Omega} \p{c} v_{ab} f^1_{abc} \left(u_{ij}, \p{k} u_{ij} \right) dx &= 0
\end{align*}
where $f^0$ and $f^1$ depend on $u$ only to first derivatives.  The pointwise functions $f^0$
and $f^1$ are exposed to petsc, with $u_{ij}$ passed as an $(i,j)$-major array and
$\p{k} u_{ij}$ as an $(i,j,k)$-major array.  $f^0_{ab}$ is split into functions $f^0_a$ each
returning a $b$-major array, and $f^1_{abc}$ is split into functions $f^1_a$ each returning
a $(b,c)$-major array.  In weird pseudocode signatures,
\begin{verbatim}
  #define AUX a[i,j], da[i,j]/dx[k], x[k]
  f0[a] : (u[i,j], du[i,j]/dx[k], AUX) -> [b];
  f1[a] : (u[i,j], du[i,j]/dx[k], AUX) -> [b,c];
\end{verbatim}
where $x_k$ is position and $a_{ij}$ are auxiliary fields.

\section{Boundary conditions}

With boundary conditions, the weak form of the PDE becomes
\begin{align*}
  \int_{\Omega} v_{ab} f^0_{ab} dx
    + \int_{\Omega} \p{c} v_{ab} f^1_{abc} dx
    + \int_{\delta \Omega} v_{ab} f^{0\p{}}_{ab} da
    + \int_{\delta \Omega} \p{c} v_{ab} f^{1\p{}}_{abc} da
   &= 0
\end{align*}
where the boundary condition terms $f^{0\p{}}$ and $f^{1\p{}}$ additionally depend on
outward pointing boundary normals $n_k$.  The pseudocode signatures are the same
as for $f^0$ and $f^1$ except for the dependence on normals:
\begin{verbatim}
  f0b[a] : (..., n[k]) -> [b]
  f1b[a] : (..., n[k]) -> [b,c]
\end{verbatim}

\section{Jacobians}

The above is sufficient for computing residuals.  For residual Jacobians, we additionally need
first derivatives of $f^0$ and $f^1$ w.r.t.\ $u$ and $\nabla u$, for a total of four additional
functions $g^0, g^1, g^2, g^3$.  Theoretically we might also need derivatives of the boundary
terms, but this is currently unsupported.

We split the 4-valued indexed $g^\gamma$ into $g^{\alpha\beta}$ where
$\gamma = 2\alpha+\beta$.  $g^{\alpha\beta}$ represents the Jacobian of $f^\alpha$ w.r.t.\
the $\beta$th derivatives of $u$:
\begin{align*}
  g^{00}_{aibj}  &= \frac{d}{du_{ij}} f^0_{ab} \\
  g^{01}_{aibjk} &= \frac{d}{d\p{k}u_{ij}} f^0_{ab} \\
  g^{10}_{aibcj} &= \frac{d}{du_{ij}} f^1_{abc} \\
  g^{11}_{aibcjk} &= \frac{d}{d\p{k}u_{ij}} f^1_{abc}
\end{align*}
Each $g^{\alpha\beta}$ is split into a two dimensional array of functions $g^{\alpha\beta}_{ai}$
giving function $f^\alpha_a$ differentiated against the $\beta$th derivatives of $u_i$.  In signatures:
\begin{verbatim}
  g^{00}[a,i] : (...) -> [b,j];
  g^{01}[a,i] : (...) -> [b,j,k];
  g^{10}[a,i] : (...) -> [b,c,j];
  g^{11}[a,i] : (...) -> [b,c,j,k];
\end{verbatim}

\end{document}
